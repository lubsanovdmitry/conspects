\documentclass[a4paper,11pt]{extarticle}

%%% Работа с русским языком
			% русские буквы в формулах
\usepackage[T1,T2A]{fontenc}			% кодировка
\usepackage[utf8]{inputenc}			% кодировка исходного текста
\usepackage[english,russian]{babel}	% локализация и переносы
%\usepackage{indentfirst}            % красная строка в первом абзаце
\frenchspacing                      % равные пробелы между словами и предложениями
\usepackage{cmap}					% поиск в PDF
\usepackage{mathtext} 	



%%% Дополнительная работа с математикой
\usepackage{amsmath,amsfonts,amssymb,amsthm,mathtools} % пакеты AMS
\usepackage{icomma}                                    % "Умная" запятая

%%% Свои символы и команды
\usepackage{centernot} % центрированное зачеркивание символа
\usepackage{stmaryrd}  % некоторые спецсимволы
\usepackage[]{mathdots}
\usepackage{pgfplots}
\pgfplotsset{compat=1.18}

\renewcommand{\epsilon}{\ensuremath{\varepsilon}}
\renewcommand{\phi}{\ensuremath{\varphi}}
\renewcommand{\kappa}{\ensuremath{\varkappa}}
\renewcommand{\le}{\ensuremath{\leqslant}}
\renewcommand{\leq}{\ensuremath{\leqslant}}
\renewcommand{\ge}{\ensuremath{\geqslant}}
\renewcommand{\geq}{\ensuremath{\geqslant}}
\renewcommand{\emptyset}{\ensuremath{\varnothing}}

\DeclareMathOperator{\sgn}{sgn}
\DeclareMathOperator{\ke}{Ker}
\DeclareMathOperator{\im}{Im}
\DeclareMathOperator{\re}{Re}
\DeclareMathOperator{\E}{E}
\DeclareMathOperator{\lif}{\rightarrow}
\DeclareMathOperator{\sm}{\setminus}
\DeclareMathOperator{\sse}{\subseteq}
\DeclareMathOperator{\sd}{\mathbin{\triangle}}
\DeclareMathOperator{\Mat}{Mat}
\DeclareMathOperator{\Ms}{M}
\DeclareMathOperator{\tr}{tr}

\newcommand{\N}{\mathbb{N}}
\newcommand{\Z}{\mathbb{Z}}
\newcommand{\Q}{\mathbb{Q}}
\newcommand{\R}{\mathbb{R}}
\newcommand{\Cm}{\mathbb{C}}
\newcommand{\F}{\mathbb{F}}
\newcommand{\id}{\mathrm{id}}
\newcommand{\ri}{\right)}
\newcommand{\li}{\left(}

\newcommand{\imp}[2]{
	(#1\,\,$\ra$\,\,#2)\,\,
}
\newcommand{\System}[1]{
	\left\{\begin{aligned}#1\end{aligned}\right.
}
\newcommand{\Root}[2]{
	\left\{\!\sqrt[#1]{#2}\right\}
}

\renewcommand\labelitemi{$\triangleright$}

\let\bs\backslash
\let\Lra\Leftrightarrow
\let\lra\leftrightarrow
\let\Ra\Rightarrow
\let\ra\rightarrow
\let\La\Leftarrow
\let\la\leftarrow
\let\emb\hookrightarrow

%%% Перенос знаков в формулах (по Львовскому)
\newcommand{\hm}[1]{#1\nobreak\discretionary{}{\hbox{$\mathsurround=0pt #1$}}{}}

%%% Работа с картинками
\usepackage{graphicx}    % Для вставки рисунков
\setlength\fboxsep{3pt}  % Отступ рамки \fbox{} от рисунка
\setlength\fboxrule{1pt} % Толщина линий рамки \fbox{}
\usepackage{wrapfig}     % Обтекание рисунков текстом

%%% Работа с таблицами
\usepackage{array,tabularx,tabulary,booktabs} % Дополнительная работа с таблицами
\usepackage{longtable}                        % Длинные таблицы
\usepackage{multirow}                         % Слияние строк в таблице

%%% Теоремы
\theoremstyle{plain}
\newtheorem{theorem}{Теорема}[section]
\newtheorem{lemma}{Лемма}[section]
\newtheorem{proposition}{Утверждение}[section]
\newtheorem*{exercise}{Упражнение}
\newtheorem{question}[]{Вопрос}


\theoremstyle{definition}
\newtheorem{definition}{Определение}[section]
\newtheorem*{corollary}{Следствие}
\newtheorem*{note}{Замечание}
\newtheorem*{reminder}{Напоминание}
\newtheorem*{example}{Пример}
\newtheorem{problem}{Задача}
\newtheorem*{problem*}{Задача}
\newtheorem*{defini}{Формулировка}

\theoremstyle{remark}
\newtheorem*{solution}{Решение}

%%% Оформление страницы
\usepackage{extsizes}     % Возможность сделать 14-й шрифт
\usepackage{geometry}     % Простой способ задавать поля
\usepackage{setspace}     % Интерлиньяж
\usepackage{enumitem}     % Настройка окружений itemize и enumerate
\setlist{leftmargin=25pt} % Отступы в itemize и enumerate

\geometry{top=25mm}    % Поля сверху страницы
\geometry{bottom=15mm} % Поля снизу страницы
\geometry{left=20mm}   % Поля слева страницы
\geometry{right=20mm}  % Поля справа страницы

\setlength\parindent{0pt}        % Устанавливает длину красной строки 15pt
\linespread{1.3}                  % Коэффициент межстрочного интервала
%\setlength{\parskip}{0.5em}      % Вертикальный интервал между абзацами
%\setcounter{secnumdepth}{0}      % Отключение нумерации разделов
%\setcounter{section}{-1}         % Нумерация секций с нуля
\usepackage{multicol}			  % Для текста в нескольких колонках
\usepackage{soulutf8}             % Модификаторы начертания

%%% Содержаниие
\usepackage{tocloft}
\tocloftpagestyle{main}
%\setlength{\cftsecnumwidth}{2.3em}
%\renewcommand{\cftsecdotsep}{1}
%\renewcommand{\cftsecpresnum}{\hfill}
%\renewcommand{\cftsecaftersnum}{\quad}

%%% Шаблонная информация для титульного листа
\newcommand{\CourseName}{Матан}
\newcommand{\FullCourseNameFirstPart}{\so{Математический Анализ}}
\newcommand{\SemesterNumber}{1}
\newcommand{\LecturerInitials}{В. В. Промыслов}
\newcommand{\CourseDate}{осень 2023}
\newcommand{\AuthorInitials}{Дмитрий Лубсанов}
\newcommand{\VKLink}{https://vk.com/dmitrylubsanov}
\newcommand{\GithubLink}{https://github.com/lubsanovdmitry}
\def\sym{\mathbin{\triangle}}
%%% Колонтитулы
\usepackage{titleps}
\newpagestyle{main}{
	\setheadrule{0.4pt}
	\sethead{\CourseName}{}{}
	\setfootrule{0.4pt}                       
	\setfoot{}{}{\thepage} 
}
\pagestyle{main}  

%%% Нумерация уравнений
%\makeatletter
%\def\eqref{\@ifstar\@eqref\@@eqref}
%\def\@eqref#1{\textup{\tagform@{\ref*{#1}}}}
%\def\@@eqref#1{\textup{\tagform@{\ref{#1}}}}
%\makeatother                      % \eqref* без гиперссылки
%\numberwithin{equation}{section}  % Нумерация вида (номер_секции).(номер_уравнения)
%\mathtoolsset{showonlyrefs=false} % Номера только у формул с \eqref{} в тексте.

%%% Гиперссылки
\usepackage{hyperref}
\usepackage[usenames,dvipsnames,svgnames,table,rgb]{xcolor}
\hypersetup{
	unicode=true,            % русские буквы в раздела PDF
	colorlinks=true,       	 % Цветные ссылки вместо ссылок в рамках
	linkcolor=black!15!blue, % Внутренние ссылки
	citecolor=green,         % Ссылки на библиографию
	filecolor=magenta,       % Ссылки на файлы
	urlcolor=NavyBlue,       % Ссылки на URL
}

%%% Графика
\usepackage{tikz}        % Графический пакет tikz
\usepackage{tikz-cd}     % Коммутативные диаграммы
\usepackage{tkz-euclide} % Геометрия
\usepackage{stackengine} % Многострочные тексты в картинках
\usetikzlibrary{angles, babel, quotes}

\DeclareMathOperator{\rank}{rank}
\makeatletter
\newenvironment{sqcases}{%
	\matrix@check\sqcases\env@sqcases
}{%
	\endarray\right.%
}
\def\env@sqcases{%
	\let\@ifnextchar\new@ifnextchar
	\left\lbrack
	\def\arraystretch{1.2}%
	\array{@{}l@{\quad}l@{}}%
}
\makeatother

\begin{document}
	\textbf{1. Сформулируйте аксиому непрерывности д/вещественных чисел.}
	
	Пусть есть 2 непустых множества $A$ и $B$ таких, что $A$ левее $B$ (т.е. $\forall a \in A \ \forall b \in B \hookrightarrow a \le b$). Тогда выполнен принцип полноты, если найдётся разделяющий их элемент, а более формально $\forall a \in A \ \forall b \in B \hookrightarrow a\le c \le b$.
	
	\textbf{2. Сформулируйте определение верхней и нижней грани, а также максимума мн-ва.}
	
	\begin{enumerate}
		\item Число $M \in \mathbb{R}$ наз-ся верхней гранью мн-ва $A \sse \mathbb{R}$, если число $M$ лежит справа от мн-ва $A$, т.е. $\forall a \in A \hookrightarrow a \le M$.
		
		\item Число $m \in \mathbb{R}$ наз-ся нижней гранью мн-ва $A \sse \mathbb{R}$, если число $m$ лежит справа от мн-ва $A$, т.е. $\forall a \in A \hookrightarrow a \ge m$.
		
		\item Число $M$ наз-ся максимальным эл-том мн-ва ($M = \max A$), если $M \in A$ и $M$ --- верхняя грань мн-ва. Аналогично для минимума.
	\end{enumerate}
	
	\textbf{3. Сформулируйте определение точной верхней и нижней граней. }
	
	Число $M$ наз-ся точной верхней (нижней) гранью/супремумом (инфинумом) мн-ва , если оно минимальная (максимальная) верхняя (нижняя) грань мн-ва (не сущ-т числа, меньшего (большего) $M$) и являющегося гранью).
	
	\textbf{4. Приведите опр-е числовой посл-ти. }
	
	Если каждому числу $n \in \N$ поставлено в соответствие какое-то число $a_n$, то говорим, что задана последовательность $ \textstyle\{a_n\}_{n=1}^{\infty}$.
	
	Формально, числовой последовательностью ${a_n}$ называется функция $a : N \rightarrow R$, где
	$a(n) = a_n$ для любого $n\in N$. Элемент последовательности - это пара $(n, an)$, где $n$ - номер
	элемента последовательности, а an - значение элемента последовательности.
	
	\textbf{5. Приведите опр-е предела Ч.П.}
	
	Пределом ч.п. наз-м такое число $a$, что для каждого $\epsilon > 0$ найдётся $N(\epsilon)$, что для всякого $n > N$ выполняется, что $a_n \in U_\epsilon(a)$ ($|a_n - a| < \epsilon$).
	
	\textbf{6. Сф-те лемму об отделимости}
	
	Если у посл-ти сущ-т предел и он не равен 0, то сущ-т $N$ такой,  что для каждого $n > N$ выполняется, что модуль члена $a_n$ больше модуля предела пополам  $\frac{|a|}{2}$ и все это больше 0.
	
	\textbf{7. Перечислите арифм. св-ва предела п-ти}
	
	\begin{enumerate}
		\item  $\lim\limits_{n\rightarrow\infty} (\alpha a_n + \beta b_n) = \alpha a + \beta b$
		\item  $\lim\limits_{n\rightarrow\infty} a_nb_n = ab$
		\item $\lim\limits_{n\rightarrow\infty} \frac{a_n}{b_n} = \frac{a}{b}$
	\end{enumerate}

	\textbf{8. Сф-те т. о переходу к пределу в неравенстве}
	
	Пусть есть две посл-ти с соотв. им пределами. Тогда если есть $N$ с которого выполнено, что элементы одной не больше эл-тов другой, то предел первой не более предела второй.
	
	\textbf{9. Сф-те лемму о зажатой посл-ти}
	
	Пусть есть две посл-ти, сходящиеся к одному пределу, и есть третья, причем такая, что сущ-т $N$, с которого выполнено $a_n \le c_n \le b_n$. Но тогда её предел равен пределу первых двух.
	
	\textbf{10. Сф-те т. Вейерштрасса о пределе} 
	
	Если посл-ть не убывает и ограничена сверху, то она сходится к точной верхней грани (и у неё сущ-т предел). Аналогично для не возрастающих.
	
	\textbf{11. Определите число $e$ }
	
	Числом $e$ является предел посл-ти $(1 + \frac{1}{n})^n$.
	
	\textbf{12. Сф-те принцип вложенных отрезков}
	
	Пусть есть такие вложенные отрезки: $[a_{n+1}; b_{n+1} ]\sse [a_n;b_n]$. Они имеют общую точку. А если их длина стремится к 0 ($\{b_n - a_n\} \rightarrow 0$), то такая точка только одна.
	
	\textbf{13. Сф-те определение подпосл-ти частичного предела}
	
	пусть задана последовательность $ \textstyle\{a_n\}_{n=1}^{\infty}$и задана возрастающая последовательность натуральных чисел $n_1 < n_2 < ...$, тогда последовательность $b_k = a_{n_k}$ называют подпоследовательностью последовательности $ \textstyle\{a_n\}_{n=1}^{\infty}$
	
	Информально, выберем из посл-ти только некоторые эл-ты.
	
	Частичным пределом, соот-но, будет предел этой подпосл-ти.
	
	\textbf{14. Определение верхнего и нижнего пределов}
	
	верхний предел --- максимальный из всех частичных пределов
	
	рассмотрим последовательности $M_n := \displaystyle\sup_{k > n}a_k, \ m_n := \displaystyle\inf_{k > n}a_k$, при этом заметим, что $M_n$ не возрастает, а $m_n$ не убывает, поэтому для ограниченной последовательности $ \textstyle\{a_n\}_{n=1}^{\infty}$ существуют пределы
	
	\[\overline{\lim_{n \rightarrow \infty}}a_n = \lim_{n \rightarrow \infty}M_n,  \ \ \ \ \underline{\lim}_{n \rightarrow \infty}a_n = \lim_{n \rightarrow \infty} m_n\]
	
	\textbf{15. Т. о том, какие значения принимает предел огр. посл-ти}
	
	Частичные пределы такой посл-ти принадлежат отрезку между нижним и верхними пределами.
	
	\textbf{16. Т. Больцано}
	
	Во всякой огр. посл-ти найдётся хоть один частичный предел
	
	\textbf{17. Сф-те критерий сходимости в терминах част. пред.}
	
	Посл-ть сходится тогда и только тогда, когда мн-во её частичных пределов состоит из одного эл-та.
	
	Обратно, если посл-ть сходится, то её частичные пределы равны пределу самой посл-ти. 
	
	\textbf{18. Опр-те фундаментальную посл-ть}
	
	Фунд. посл-ть это такая посл-ть, что найдется такой $N$, что после него разница между любыми двумя членами меньше $\epsilon$.
	
	\textbf{19.  Критерий Коши}
	
	Если посл-ть сходится, она фундаментальна.
	
	Обратно, если посл-ть фундаментальна, то на сходится. 
	
	\textbf{20. Что такое числовой ряд}
	
	Пусть есть некая посл-ть. Тогда числ. рядом будем сумма членов этой посл-ти.
	
	\textbf{21. Критерий Коши для числ. рядов}
	
	Ряд сходится т. и т. т., когда начиная с некого $N$ разница  всех частичных сумм ряда меньше $\epsilon$.
	
	\textbf{22. Необходимое условие сходимости ряда}
	
	Если ряд сходится, то посл-ть стремится к 0.
	
	\textbf{23.  Что такое гармонический ряд? Сходится ли он?}
	
	Гарм. ряд это ряд  $\displaystyle \sum_{n=1}^{\infty}\frac1n$ и он не сходится.
	
	\textbf{24. Опр-те абс. и усл. сходимость ряда	}
	
	Если ряд $\displaystyle \sum_{n=1}^{\infty} |a_k|$ сходится, то $\displaystyle \sum_{n=1}^{\infty} a_k$ сходится абсолютно. Иначе он сходится условно.
	
	\textbf{25. Т. об огранич. частичных сумм сходящегося ряда с неотр. членами}
	
	пусть $a_k \geq 0 \ \forall k \in \N$, тогда ряд $\displaystyle\sum_{k = m + 1}^{n}a_k$ сходится тогда и только тогда, когда последовательность его частичных сумм ограничена
	
	\textbf{26. Признак сравнения числ. рядов}
	
	пусть $0 \le a_n \le b_n$, тогда если ряд $\displaystyle\sum_{k = 1}^{\infty}b_k$ сходится, то и $\displaystyle\sum_{k = 1}^{\infty}a_k$ сходится; если же $\displaystyle\sum_{k = 1}^{\infty}a_k$  расходится, то и $\displaystyle\sum_{k = 1}^{\infty}b_k$ расходится
	
	\textbf{27. Признак Коши}
	
	пусть $ \textstyle\{a_n\}_{n=1}^{\infty}$ - невозрастающая последовательность, $a_n \geq 0$. ряд $\displaystyle\sum_{k = 1}^{\infty}a_k$ сходится тогда и только тогда, когда сходится ряд $\displaystyle\sum_{k = 1}^{\infty}2^ka_{2^k}$
	
	\textbf{28. Про ряд $\displaystyle \sum_{k=1}^{\infty} \frac{1}{k^p}$}
	
	такой ряд сходится при $p > 1$ и расходится при $p \leq 1$
	
	\textbf{29. Перестановка слагаемых в рядах}
	
	Ряд получен перестановкой, если сущ-т посл-ть нат. чисел, задающая биективное преобразование $\mathbb{N}$  и такая, что для всякого $n$ $\tilde{a}_n = a_{k_j}$.	
	
	\textbf{30. Т. Римана}
	
	
	\textbf{37. I замечательный предел}
	
	$\displaystyle \lim\limits_{n\rightarrow0} \frac{\sin x}{x} = 1$
	
	\textbf{38. II замечательный предел}
	
	$\displaystyle \lim\limits_{n\rightarrow+\infty}  \li 1 + \frac{1}{x}\ri^x = e$
	
\end{document}